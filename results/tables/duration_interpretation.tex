\paragraph{Duration Adjustment Impact.}
Table~\ref{tab:duration_sensitivity} demonstrates the impact of duration assumptions
on SCR and aggregate ASRI. Moving from short-duration reserves (T-bills, duration 0.25 years)
to long-duration positions (5-year T-notes) increases peak SCR by 4.3 points
and peak ASRI by 0.6 points. Critically, detection rates remain unchanged
across all duration scenarios---all four crises are detected at the operational threshold
($\tau=50$) regardless of duration assumption.

This robustness arises because duration adjustment scales the Treasury component
proportionally, preserving relative rankings during stress periods. The SVB crisis,
which specifically involved duration mismatch (banks holding long-duration securities
funded by short-term deposits), is detected under all specifications, validating that
the current implementation captures the relevant risk dynamics even without explicit
duration modeling.

For practitioners with specific knowledge of stablecoin reserve compositions,
duration-adjusted SCR provides a more accurate risk estimate. The baseline implementation
conservatively assumes short-duration reserves, which slightly underestimates risk for
issuers with longer-duration holdings. Future versions will incorporate issuer-specific
duration data as attestation reports improve disclosure standards.