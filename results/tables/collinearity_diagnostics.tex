\begin{table}[H]
\centering
\caption{Collinearity Diagnostics for Sub-Indices}
\label{tab:collinearity_diagnostics}
\small
\begin{tabular}{lcc}
\toprule
Diagnostic & Value & Interpretation \\
\midrule
\multicolumn{3}{l}{\textit{Variance Inflation Factors}} \\
VIF(SCR) & 3.39 & Low collinearity \\
VIF(DLR) & 3.67 & Low collinearity \\
VIF(CR) & 3.89 & Low collinearity \\
VIF(AO) & 3.03 & Low collinearity \\
\midrule
\multicolumn{3}{l}{\textit{Principal Component Analysis}} \\
PC1 variance explained & 59.1\% & Cumulative: 59.1\% \\
PC2 variance explained & 32.4\% & Cumulative: 91.6\% \\
PC3 variance explained & 5.3\% & Cumulative: 96.9\% \\
PC4 variance explained & 3.1\% & Cumulative: 100.0\% \\
\midrule
\multicolumn{3}{l}{\textit{Matrix Diagnostics}} \\
Condition number & 19.1 & Weak collinearity \\
Max eigenvalue & 2.366 & -- \\
Min eigenvalue & 0.124 & Ratio = 19.1 \\
\bottomrule
\end{tabular}
\begin{tablenotes}
\small
\item VIF $< 5$: acceptable; VIF $> 10$: problematic.
\item Condition number $< 30$: weak collinearity.
\item All 4 PCs required indicates sub-indices capture distinct variance.
\end{tablenotes}
\end{table}