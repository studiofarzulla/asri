\documentclass[11pt]{letter}
\usepackage[margin=1in]{geometry}
\usepackage{hyperref}

\signature{Murad Farzulla\\King's College London / Farzulla Research\\murad.farzulla@kcl.ac.uk\\murad@farzulla.org}
\address{}

\begin{document}

\begin{letter}{Digital Finance\\Editorial Office}

\opening{Dear Editors,}

We are pleased to submit the manuscript ``ASRI: An Aggregated Systemic Risk Index for Cryptocurrency Markets'' for consideration in Digital Finance. This paper is co-authored with Andrew Maksakov (King's College London).

This study introduces ASRI, the first composite systemic risk index designed specifically for cryptocurrency and DeFi markets. Aggregating four sub-indices---Stablecoin Risk, DeFi Liquidity Risk, Contagion Risk, and Opacity Risk---into a daily composite measure, the index provides a unified framework for monitoring DeFi-TradFi interconnection risk. Applied to the period January 2021 to December 2024, ASRI detects all four major crypto crises (Terra/Luna, Celsius/3AC, FTX, SVB) with statistically significant cumulative abnormal stress ($p < 0.001$ for all events after Bonferroni correction) and a mean lead time of 18 days before crisis onset.

Key contributions include: (1) an axiomatic construction grounding the index in formal risk properties (monotonicity, sub-additivity, tail sensitivity), (2) empirical validation via bootstrap detection analysis achieving 3/4 threshold detection and 4/4 event study significance, (3) a Hidden Markov Model regime analysis identifying distinct low-risk, moderate, elevated, and crisis states, and (4) a fully open-source implementation with a live monitoring dashboard at \url{https://asri.dissensus.ai}.

We believe this work aligns well with Digital Finance's scope at the intersection of systemic risk measurement, DeFi market microstructure, and computational finance methodology. The manuscript has not been submitted elsewhere. Full replication code is available at \url{https://github.com/studiofarzulla/asri} under MIT License.

\closing{Sincerely,}

\end{letter}
\end{document}
